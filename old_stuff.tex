% \COMMENT{Christine Nattrass comments:  
% RHIC:  ALICE, STAR and PHENIX now require \hepdata with publication now. I proposed it for sPHENIX; it was decided that the mechanism for how the data points were distributed would be a procedure, not in the publication policy, but as a practical matter, I think \hepdata will be adopted by sPHENIX. Phobos and BRAHMS do not officially have a procedure to get data up. BRAHMS has some effort which is less formal. Phobos only has two papers up on \hepdata. (Note Phobos is not an acronym, hence not being all caps.)  That is, overall, the heavy ion community has adopted \hepdata - but we still have issues with previously published papers.

% For previously published papers:  STAR made formatting data from old papers for \hepdata an alternative to shifts during the pandemic.

% In PHENIX, I just put in a proposal to hire undergrads to do this. Here's some text out of that proposal, with light editing for coherence:

% \hrulex
% \hepdata only became obligatory for STAR in 2019 and PHENIX in 2020, after a proposal by Nattrass to the PHENIX Executive Council. Around this time, both experiments' web sites suffered prolonged outages due to cybersecurity incidents, making uploading data to \hepdata a much higher priority. In early 2020, it was determined that the old PHENIX web site hosting data would no longer be externally available.  \hepdata is now the \textit{only} means of disseminating PHENIX data. Nattrass helped develop a procedure for submission of PHENIX data to \hepdata, is helping oversee the submission of PHENIX data to \hepdata (including previously published data), and is assisting primary authors with formatting data. Data from previously published papers, however, are still generally unavailable outside of the collaboration.

% As of January~12, 2022, there are 224 published PHENIX papers and 55 PHENIX papers (25\%) with data available on \hepdata. The majority of papers without data available on \hepdata are earlier papers, including many seminal results in the field. The previously-public PHENIX web page where the data were is still available to PHENIX collaborators internally, but there was no enforced structure. The varied nature of the papers, as well as some issues with the clarity or completeness of  these data, means that there is no trivial, automated way to convert the data to the format required for submission to \hepdata. Furthermore, the preparation of the YAML files for \hepdata requires some knowledge of high energy physics. This task is currently being done voluntarily by PHENIX collaborators, who must learn how to format data for \hepdata, and parasitically by undergraduates working on \rivet. It is therefore very slow.

% Nattrass has supervised several undergraduates as they implement this process, including several with limited programming experience. We estimate that it takes between 2--20 hours for an experienced worker to format data for \hepdata, depending on the complexity of the paper. The simplest papers may only have one or two data points, while at least one paper formatted by our group had over 2000 data tables with about 20 data points each. A beginning undergraduate would need to learn some basics of the field, and could use the materials developed for the CURE. We estimate this would take approximately 45 hours, commensurate with the time it takes in the CURE, and that we could complete approximately 100 papers by hiring two undergraduates part time during the semester and full time in the summer. This task is scalable, so if less funding is available, it would still be possible to format some PHENIX data for upload to \hepdata.
% \hrulex

% Successfully uploading data to \hepdata also requires extensive communication with and contributions from current and former PHENIX collaborators, both to resolve questions about the data and to vet the final product before it is made public. Most people are incredibly enthusiastic to have someone else format data from a measurement they worked on, and the collaboration membership and leadership strongly support this effort. We have not had problems getting assistance in the past and do not anticipate problems in the future.
% }
\documentclass[11pt]{article}

\PassOptionsToPackage{x11names,svgnames}{xcolor}
\usepackage[utf8]{inputenc}
\usepackage{cite}  % in text chain sequential citations in bibliography with a dash
\usepackage{graphicx}
\usepackage[margin=1.25in]{geometry}
\usepackage{xcolor}
\usepackage{tcolorbox}
%\usepackage[usenames,dvipsnames]{xcolor}
\usepackage{url}
\usepackage[colorlinks = true,
            linkcolor = blue,
            urlcolor  = blue,
            citecolor = blue,
            anchorcolor = blue]{hyperref}
            
\usepackage{xspace}
\usepackage{orcidlink}
\usepackage{lineno}
\usepackage{todonotes}
\usepackage[toc]{glossaries}
\usepackage{cleveref}
\usepackage[page,title,toc]{appendix}

% basic data for the eprint:

\textwidth=6.0in  \textheight=8.5in

% Adjust these for your printer:
\parskip=0.1truein 

% preprint number data:
\newcommand\pubnumber{DRAFT}
\newcommand\pubdate{\today}

%   document style macros
\def\Title#1{\begin{center} {\LARGE #1 } \end{center}}
\def\Author#1{\begin{center}{ \sc #1} \end{center}}
\def\Address#1{\begin{center}{ \it #1} \end{center}}
\def\andauth{\begin{center}{and} \end{center}}
\def\submit#1{\begin{center}Submitted to {\sl #1} \end{center}}
\newcommand\pubblock{\rightline{\begin{tabular}{l} \pubnumber\\
         \pubdate \end{tabular}}}
\newenvironment{Abstract}{\begin{quotation} \begin{center}
                       ABSTRACT
     \end{center}\bigskip  }{\end{quotation}}
\newenvironment{Presented}{\begin{quotation} \begin{center} 
             CONTRIBUTED TO\end{center}\bigskip 
      \begin{center}\begin{large}}{\end{large}\end{center} \end{quotation}}
\def\submit#1{\begin{center}Submitted to {\sl #1} \end{center}}
\def\Acknowledgements{\bigskip  \bigskip \begin{center} \begin{large}
             \bf ACKNOWLEDGEMENTS \end{large}\end{center}}

\newcommand\snowmass{\begin{center}\rule[-0.2in]{\hsize}{0.01in}\\\rule{\hsize}{0.01in}\\
\vskip 0.1in Submitted to the  Proceedings of the US Community Study\\ 
on the Future of Particle Physics (Snowmass 2021)\\ 
\rule{\hsize}{0.01in}\\\rule[+0.2in]{\hsize}{0.01in} \end{center}}

%  personal abbreviations and macros

\newcommand{\hrulex}{%
  \vspace{1ex}
  \hrule
  \vspace{1ex}}

\newcommand\NB[1]{\textbf{N.B.}: {#1}}

\newcommand{\TODO}[2][]{\todo[inline,#1]{\textbf{TODO:} #2}}
\newcommand{\COMMENT}[2][]{\todo[inline,color=DarkOliveGreen1!90!Gray,#1]{#2}}
\newcommand{\ALERT}[2][]{\todo[inline,color=Red2!60!Orange!90,#1]{\textbf{!!:} #2}}
\newcommand{\QUESTION}[2][]{\todo[inline,color=SkyBlue1!90!Gray,#1]{\textbf{Q:} #2}}
\newcommand{\EDITOR}[2][]{\todo[inline,color=Green!60!Orange!90,#1]{\textbf{!!:} #2}}


\newcommand{\rivet}{Rivet\xspace} 
\newcommand{\contur}{Contur\xspace} 
\newcommand{\hepdata}{HEPData\xspace} 
\newcommand{\recast}{\textsc{Recast}\xspace} 
\newcommand{\madanalysis}{MadAnalysis5\xspace} 
\newcommand{\checkmate}{CheckMATE\xspace} 
\newcommand{\smodels}{SModelS\xspace} 
\newcommand{\gambit}{GAMBIT\xspace} 
\newcommand{\colliderbit}{ColliderBit\xspace} 
\newcommand{\simpleanalysis}{SimpleAnalysis\xspace} 
\newcommand{\delphes}{\textsc{Delphes}\xspace} 


%%%%   Standard symbols for use in Snowmass Working Group Proceedings%%%%%%%%%%%%%%%%%%%%%%%%%%%%%%%%%%%%%%%%%%%%%%%%%%%%%%%%%%%%%%%%%%%%%%%%%%%  basic formatting macros:%%%%%%%%%%%%%%%%%%%%%%%%%%%%%%%%%%%%%%%%%%%%%%%%%%%%%%%%%%%%%%%%%%%


%   text mode macros

\def\ie{{\it i.e.}}
\def\eg{{\it e.g.}}
\def\etc{{\it etc.}}
\def\etal{{\it et al.}}
\def\ibid{{\it ibid}.}

\def\Acknowledgements{\bigskip  \bigskip \begin{center} \begin{large}
             \bf ACKNOWLEDGEMENTS \end{large}\end{center}}


%  equation display macros

\def\beq{\begin{equation}}
\def\eeq#1{\label{#1}\end{equation}}
\def\eeqn{\end{equation}}

%%  multiple-line equations  (use \CR as the carriage return):

\newenvironment{Eqnarray}%
   {\arraycolsep 0.14em\begin{eqnarray}}{\end{eqnarray}}
\def\beqa{\begin{Eqnarray}}
\def\eeqa#1{\label{#1}\end{Eqnarray}}
\def\eeqan{\end{Eqnarray}}
\def\CR{\nonumber \\ }

%%  reference to an equation number:

\def\leqn#1{(\ref{#1})}

%%  sizing and bars

\def\st{\scriptstyle}
\def\sst{\scriptscriptstyle}

\def\overbar#1{\overline{#1}}
\let\littlebar=\bar
\let\bar=\overbar

\def\subscr#1{{\mbox{\scriptsize #1}}}



%%%%%%%%%%%%%%%%%%%%%%%%%%%%%%%%%%%%%%%%%%%%%%%%%%%%%%%%%%%%%%%%%%%%%%%%%%

%%  expectation values: 

\def\VEV#1{\left\langle{ #1} \right\rangle}
\def\bra#1{\left\langle{ #1} \right|}
\def\ket#1{\left| {#1} \right\rangle}
\def\vev#1{\langle #1 \rangle}

%%%%%%%%%%%%%%%%%%%%%%%%%%%%%%%%%%%%%%%%%%%%%%%%%%%%%%%%%%%%%%%%%%%%%%%%%

%% relation symbols

\def\lsim{\mathrel{\raise.3ex\hbox{$<$\kern-.75em\lower1ex\hbox{$\sim$}}}}
\def\gsim{\mathrel{\raise.3ex\hbox{$>$\kern-.75em\lower1ex\hbox{$\sim$}}}}

\def\Im{{\rm Im}}
\def\Re{{\rm Re}}
\def\tr{{ \rm tr}}

%%%%%%%%%%%%%%%%%%%%%%%%%%%%%%%%%%%%%%%%%%%%%%%%%%%%%%%%%%%%%%%%%%%%%%%%%%%%%%
%%  caligraphic letters (for matrix elements, luminosity, etc.)

\def\D{{\cal D}}
\def\L{{\cal L}}
\def\M{{\cal M}}
\def\O{{\cal O}}
\def\W{{\cal W}}
\def\H{{\cal H}}

%%%%%%%%%%%%%%%%%%%%%%%%%%%%%%%%%%%%%%%%%%%%%%%%%%%%%%%%%%%%%%%%%%%%%

%%  matrix operations and fractions:

\def\One{{\bf 1}}
\def\hc{{\mbox{\rm h.c.}}}
\def\tr{{\mbox{\rm tr}}}
\def\half{\frac{1}{2}}
\def\thalf{\frac{3}{2}}
\def\third{\frac{1}{3}}
\def\tthird{\frac{2}{3}}

\def\del{\partial}
\def\Dslash{\not{\hbox{\kern-4pt $D$}}}
\def\dslash{\not{\hbox{\kern-2pt $\del$}}}
\def\pslash{\not{\hbox{\kern-2pt $p$}}}
\def\ETmiss{\not{\hbox{\kern-4pt $E$}}_T}

\def\Dlr{\mathrel{\raise1.5ex\hbox{$\leftrightarrow$\kern-1em\lower1.5ex\hbox{$D$}}}}

%%%%%%%%%%%%%%%%%%%%%%%%%%%%%%%%%%%%%%%%%%%%%%%%%%%%%%%%%%%%%%%%%%%%%%%%%%%%%

%%  high-energy physics terminology:

\def\Pl{{\mbox{\scriptsize Pl}}}
\def\eff{{\mbox{\scriptsize eff}}}
\def\CM{{\mbox{\scriptsize CM}}}
\def\GUT{{\mbox{\scriptsize GUT}}}
\def\BR{\mbox{\rm BR}}
\def\ee{e^+e^-}
\def\sstw{\sin^2\theta_w}
\def\cstw{\cos^2\theta_w}
\def\mz{m_Z}
\def\gz{\Gamma_Z}
\def\mw{m_W}
\def\gw{\Gamma_W}
\def\mt{m_t}
\def\gt{\Gamma_t}
\def\mh{m_H}
\def\gh{\Gamma_H}
\def\gmu{G_\mu}
\def\GF{G_F}
\def\alphas{\alpha_s}
\def\MSB{{\bar{M \kern -2pt S}}}
\def\msb{{\bar{\scriptsize M \kern -1pt S}}}
\def\lmsb{\Lambda_{\msb}}
\def\drb{{\bar{\scriptsize D \kern -1pt R}}}
\def\ELER{e^-_Le^+_R}
\def\EREL{e^-_Re^+_L}
\def\ELEL{e^-_Le^+_L}
\def\ERER{e^-_Re^+_R}
\def\eps{\epsilon}
\def\Kzero{K^0}
\def\Kzerobar{{\bar K}^0}
\def\Dzero{D^0}
\def\Dzerobar{{\bar D}^0}
\def\Bzero{B^0}
\def\Bzerobar{{\bar B}^0}


%%%%%%%%%%%%   units for math mode  %%%%%%%%%%%%%%%%%%%%%%%%%

\def\ifb{{\rm fb}^{-1}}
\def\eV{{\rm eV}}
\def\keV{{\rm keV}}
\def\MeV{{\rm MeV}}
\def\GeV{{\rm GeV}}
\def\TeV{{\rm TeV}}


%%%%%%%%%%%%%%%%%%%%%%%%%%%%%%%%%%%%%%%%%%%%%%%%%%%%%%%%%%%%%%%%%%%%%

%%  supersymmetry:

\def\ch#1{\widetilde\chi^+_{#1}}
\def\chm#1{\widetilde\chi^-_{#1}}
\def\neu#1{\widetilde\chi^0_{#1}}
\def\s#1{\widetilde{#1}}



%%%%%%%%%%  common journal names   %%%%%%%%%%%%%%%%%%%%%%%%%%%%%

\newcommand\aj{{Astronomical Journal}}%
\newcommand\araa{{Annual Review of Astron and Astrophys}}%
 \newcommand\apj{{Astrophysical Journal}}%
\newcommand\apjl{{Astrophysical Journal, Letters}}%
\newcommand\apjs{{ApJS}}%
% Astrophysical Journal, Supplement
\newcommand\ao{{Appl.~Opt.}}%
% Applied Optics
\newcommand\apss{{Ap\&SS}}%
% Astrophysics and Space Science
\newcommand\aap{{Astronomy and Astrophysics}}%
\newcommand\aapr{{A\&A~Rev.}}%
% Astronomy and Astrophysics Reviews
\newcommand\aaps{{A\&AS}}%
% Astronomy and Astrophysics, Supplement
\newcommand\azh{{AZh}}%
% Astronomicheskii Zhurnal
\newcommand\baas{{BAAS}}%
% Bulletin of the AAS
\newcommand\jrasc{{JRASC}}%
% Journal of the RAS of Canada
\newcommand\memras{{MmRAS}}%
% Memoirs of the RAS
\newcommand\mnras{{Monthly Notices of the Royal Astronomical Society}}%
\newcommand\pra{{Phys.~Rev.~A}}%
% Physical Review A: General Physics
\newcommand\prb{{Phys.~Rev.~B}}%
% Physical Review B: Solid State
\newcommand\prc{{Phys.~Rev.~C}}%
% Physical Review C
\newcommand\prd{{Physics~Review~D}}%
% Physical Review D
\newcommand\pre{{Phys.~Rev.~E}}%
% Physical Review E
\newcommand\prl{{Phys.~Rev.~Lett.}}%
% Physical Review Letters
\newcommand\pasp{{Publications of the ASP}}%
\newcommand\pasj{{PASJ}}%
% Publications of the ASJ
\newcommand\qjras{{QJRAS}}%
% Quarterly Journal of the RAS
\newcommand\jcap{{JCAP}}%
\newcommand\skytel{{S\&T}}%
% Sky and Telescope
\newcommand\solphys{{Sol.~Phys.}}%
% Solar Physics
\newcommand\sovast{{Soviet~Ast.}}%
% Soviet Astronomy
\newcommand\ssr{{Space~Sci.~Rev.}}%
% Space Science Reviews
\newcommand\zap{{ZAp}}%
% Zeitschrift fuer Astrophysik
\newcommand\nat{{Nature}}%
\newcommand\iaucirc{{IAU~Circ.}}%
% IAU Cirulars
\newcommand\aplett{{Astrophys.~Lett.}}%
% Astrophysics Letters
\newcommand\apspr{{Astrophys.~Space~Phys.~Res.}}%
% Astrophysics Space Physics Research
\newcommand\bain{{Bull.~Astron.~Inst.~Netherlands}}%
% Bulletin Astronomical Institute of the Netherlands
\newcommand\fcp{{Fund.~Cosmic~Phys.}}%
% Fundamental Cosmic Physics
\newcommand\gca{{Geochim.~Cosmochim.~Acta}}%
% Geochimica Cosmochimica Acta
\newcommand\grl{{Geophys.~Res.~Lett.}}%
% Geophysics Research Letters
\newcommand\jcp{{J.~Chem.~Phys.}}%
% Journal of Chemical Physics
\newcommand\jgr{{J.~Geophys.~Res.}}%
% Journal of Geophysics Research
\newcommand\jqsrt{{J.~Quant.~Spec.~Radiat.~Transf.}}%
% Journal of Quantitative Spectroscopy and Radiative Transfer
\newcommand\memsai{{Mem.~Soc.~Astron.~Italiana}}%
% Mem. Societa Astronomica Italiana
\newcommand\nphysa{{Nucl.~Phys.~A}}%
% Nuclear Physics A
\newcommand\physrep{{Phys.~Rep.}}%
% Physics Reports
\newcommand\physscr{{Phys.~Scr}}%
% Physica Scripta
\newcommand\planss{{Planet.~Space~Sci.}}%
% Planetary Space Science
\newcommand\procspie{{Proc.~SPIE}}%

%%%%%%%%%%%%%%%%%%%%%%%%%%%%%%%%%%%%%%

% Toggle commenting to disallow/allow line numbers
\linenumbers

% DEADLINES
% 15 January 2022: Please provide an abstract and short outline of your white paper contribution to TF07 conveners
% 23-25 February 2022: Theory Frontier Conference at the KITP (waiting for further notice from KITP)
% 15 March 2022: contributed papers (``white papers”) due 
% 31 May 2022: draft topical group reports - release for public comments 
% 30 June 2022: draft Frontier report - release for public comments 
% 17-27 July 2022: Seattle meeting (``Snowmass Community Study”)
% 30 September 2022: final reports of TGs and Frontiers
% 31 October 2022: Snowmass book including high level executive summaries online

% N.B.:
% CompF7 scope, mandate, questions:
% **White papers are encouraged to touch on these topics if applicable**
% * Functional areas
%    - Public data (comes in many forms … HepData, public likelihoods, CERN OpenData, data for education/outreach)
%       + Tools for generating annotated public data and software
%    - Tools for combining results across experiments and frontiers
%    - Tools for archiving and re-running the analysis (RECAST/REANA)
% * Mandate
%   - Define the stakeholders and consumers of the data and software
%   - What are the needs/requirements of the stakeholders?
%   - What resources are needed?
%       + e.g. long-term storage with external access, infrastructure for preserving executable code, etc. 
%       + metadata infrastructure
%   - What technologies are available or will be available, what is the technology evolution of these tools?
%       + Discussed in common with CompF5: End User Analysis:
%          * version control
%          * Containers/VMs
%          * proprietary software/licenses
%   - How are/will the stakeholders use these technologies?
%   - How are other science domains handling this topic?
%   - What are the workflows that are used to combine results across experiments and frontiers?
%   - What tools are used/needed by the stakeholders to combine results across experiments and frontiers?
%   - What is the technology evolution of these tools?
%   - What are other science domains using, what is industry using?







\begin{document}

\pubblock

\snowmass{}

\Title{\textbf{Data and Analysis Preservation, \\Recasting, and Reinterpretation}\\
\vspace{10pt}%
TF07 (Collider Phenomenology in the Theory Frontier)\\
\vspace{10pt}%
COMPF7 (Reinterpretation and long-term preservation of data and code)}

\medskip

% IF YOU DON'T SEE YOURSELF ON THE AUTHOR LIST ADD YOURSELF
% ALSO, IF YOU'D LIKE, PLEASE ADD YOUR ORCID TOO

% Author List
\begin{center}
% Stephen Bailey\,\orcidlink{0000-XXXX-XXXX-XXXX}\textsuperscript{1},
Stephen Bailey\textsuperscript{1},
Kyle Cranmer\,\orcidlink{0000-0002-5769-7094}\textsuperscript{2},
Nishita Desai\,\orcidlink{0000-0001-7942-1649}\textsuperscript{3},
Matthew Feickert\,\orcidlink{0000-0003-4124-7862}\textsuperscript{4},
Lukas Heinrich\,\orcidlink{0000-0002-4048-7584}\textsuperscript{5},
Axel Huebl\,\orcidlink{0000-0003-1943-7141}\textsuperscript{1},
Matias Carrasco Kind\,\orcidlink{0000-0002-4802-3194}\textsuperscript{4},
Sabine Kraml\,\orcidlink{0000-0002-2613-7000}\textsuperscript{6},
Clemens Lange\,\orcidlink{0000-0002-3632-3157}\textsuperscript{7},
Andre Lessa\,\orcidlink{0000-0002-5251-7891}\textsuperscript{8},
Kati Lassila-Perini\,\orcidlink{0000-0002-5502-1795}\textsuperscript{9},
Mark S. Neubauer\,\orcidlink{0000-0001-8434-9274}\textsuperscript{4},
Harrison B. Prosper\,\orcidlink{0000-0002-4077-2713}\textsuperscript{10},
Sezen Sekmen\,\orcidlink{0000-0003-1726-5681}\textsuperscript{11},
Giordon Stark\,\orcidlink{0000-0001-6616-3433}\textsuperscript{12},
Wolfgang Waltenberger\,\orcidlink{0000-0002-6215-7228}\textsuperscript{13}
\end{center}

% Affiliations 
\begin{center}
\textbf{1}~Lawrence Berkeley National Laboratory, USA
\textbf{2}~New York University, USA
\textbf{3}~Tata Institute of Fundamental Research, India
\textbf{4}~University of Illinois at Urbana-Champaign, USA
\textbf{5}~Technische Universität München, Germany
\textbf{6}~LPSC Grenoble, France
\textbf{7}~Paul Scherrer Institute, Villigen, Switzerland
\textbf{8}~Universidade Federal do ABC, Brazil
\textbf{9}~Helsinki Institute of Physics, Finland
\textbf{10}~Florida State University, USA
\textbf{11}~Kyungpook National University, Korea
\textbf{12}~SCIPP, UC Santa Cruz, CA, USA
\textbf{13}~HEPHY and University of Vienna, Austria
\end{center}

\begin{Abstract}
\noindent We make the case for the systematic, reliable preservation of data, data products, and analysis code for their long-term future reuse, in order to make the most out of particle physics experiments.
We cover the motivating needs for both the experimental and theoretical particle physics community, and outline the goals and benefits that are uniquely enabled by analysis recasting and reinterpretation. 
We discuss the technical challenges and infrastructure needs, as well as sociological challenges and changes, and give summary recommendations to the particle physics community.
\end{Abstract}
% Sabine @ Kyle et al: your may want to improve :-)

\clearpage

%-------------------------------------------------------------
\section{Introduction}
%-------------------------------------------------------------

What is needed to enable re-analysis and reuse of analyses from the LHC experiments and other collaborations to explore not-yet-thought-of theories in the long term future (i.e., on the 10 to 50 year time scale).
This requires the development and adoption of techniques and standards for long term storage and preservation of analysis data and data products, reproducible workflows for recasting and reinterpretation of analysis data products, and large sociological changes in the broader particle physics community.\\
Technical considerations along the lines of \cite{LHCReinterpretationForum:2020xtr,Cranmer:2021urp}, etc. \\

Q: To what extend do we discuss differences between high- and low-energy experiments (energy vs.\ intensity frontier), particle and astro-particle communities, etc.? 


%-------------------------------------------------------------
\section{Data preservation (open data)}
%-------------------------------------------------------------

[Status, what data is currently open (real and Monte Carlo), future plans by collaborations, FAIR Open Data, infrastructure needs, ...]

%NOTE (from Kati): this is getting too long, and it is not even done. Let me know if I should cut it down.
%Reply (from Sabine): i.m.o.\ this is a very important section; it merits a couple of pages. I would not cut it down, not at this stage. 
% OK, thanks. I'm done. It is CMS centric as usual but I guess it is justified. Feel free to adapt.


Following the positive experience and feedback from the open data releases by the CMS experiment since 2014, a CERN open data policy for the LHC experiments~\cite{cern-data-policy} was formulated in 2020. All LHC experiments are committed to releasing research-quality data through the CERN open data portal~\cite{CODP}. The amount of data and release timeline varies from an experiment to another and they are defined in the experiment-specific data policies~\cite{cern-data-policy,cern-open-data-privacy-policy,cms-open-data-policy,atlas-open-data-policy,lhcb-open-data-policy,alice-open-data-policy}. CERN as the host laboratory maintains and develops the infrastructure needed for the portal, provides storage resources and takes the custodial responsibility of released open data in long term. 

At the time of writing, CMS is the only experiment having released research-quality data. The CMS open data releases contain full reprocessing of collision data from each data-taking period and the simulated data corresponding to these data. They are made available in the format and with the same data quality requirements that analyses of the CMS collaboration start from. The volume of data, collision and Monte Carlo simulation, amounts currently to 2.8 PB. The public data are accompanied by a compatible version of the CMSSW software and additional information necessary to perform a research-level physics analysis. Example code and some specific guide pages are provided to explain and instruct the use of the associated information.

Data released through the CERN open data portal satisfy Findable, Accessible, Interoperable, and Reusable (FAIR) principles for scientific data management~\cite{FAIR-paper} for data and metadata to a large extent. Due to the complexity of experimental particle physics data, the FAIR principles alone do not guarantee the re-usability of these data, and additional effort is needed to pass on the knowledge needed to use and interpret them correctly. In large experiments, collaboration members benefit from the existing knowledge infrastructure (meetings, discussion forums, mentoring, specific groups responsible for providing data assets needed for physics analysis, etc) that is not available to external users or in long term. 

%[more details can be added here about challenges related to computing and software, and those related to intricacies of experimental particle physics data]

The ensure that the open data can really be used in physics research and that all necessary additional information is captured, stored and provided, early testing is of utmost importance. Therefore, CMS puts emphasis on releasing data relatively early after the data-taking. Feedback and questions from external users indicate the eventual missing data assets and lacking information, and the release can be complemented with them while the expertise on these data is still present in the collaboration. To encourage the use of open data and to get direct feedback, the CMS open data group is organising regular workshops with hands-on tutorials. It is important to reach out to a diverse user community, from newcomers to seniors and with different computing backgrounds and physics skills. While the CMS open data group is looking forward to being able to organize the next workshop in presence, the value of remote, virtual events with many participants who would not have been able to travel is acknowledged.

Setting up example analysis workflows is one of the priorities for making open data reusable. They should demonstrate all necessary steps from data access and selection to any eventual corrections, and address the particularities of experimental data such as estimating efficiencies and uncertainties. Activities for preserving analysis workflows internally in the collaboration will facilitate building such workflows. The software container technology, in use with open data, is well adapted for automating workflows and experience gained with them can be used for procedures for current analyses within the collaboration.

%[input from OD users for these workflows, i.e. they may have already repeated an existing  measurement with OD as a test]

%Can someone check this please:
There are many successful data preservation efforts in the LEP and HERA experiments, but no public access to these data has yet been provided in terms defined in the CERN data policy for the LHC experiments. Access to preserved data is made possible through different modalities, such as joining the collaboration, or working with collaboration members. The BABAR collaboration has decided to make its data available for future analyses through the CERN Open Data portal. Lack of personpower is often a bottleneck, and while true for experiments in the data-taking phase, it is even more evident for experiments past data taking. Open access initiatives have to start at an early stage.

%[OD beyond the LHC?  Reports in DPHEP workshop https://indico.cern.ch/event/1043155/timetable/ BaBar decision in p 10 of https://indico.cern.ch/event/1043155/contributions/4383112/attachments/2267926/3850947/BaBar-DPHEP-20210621.pdf#page=10 Difficulty of policy making after active data taking? ]

To make open data initiatives a success, building community of users for these data is paramount. Theory community is a key player in this, and promoting the use of open data despite the substantial work required in their analysis is a strong demonstration of their value on one hand, and an invaluable validation effort for their usability for future generations on the other hand.

%-------------------------------------------------------------
\section{Analysis preservation}
%-------------------------------------------------------------

[scope, status, difficulties met, person power and infrastructure needs, ....]

\subsection{Collaboration-internal preservation}

\subsection{Public information, ADL}

%-------------------------------------------------------------
\section{Preservation of data products}
%-------------------------------------------------------------

[What data products are needed? In what form? How? 
What are the challenges? (often sociological) What infrastructure need beyond current HEPData setup?]

\begin{itemize}
    \item Full probability models of analyses from experiments
    \item Simplified probability models of analyses from experiments
    \item HEPData long term support
    \item \ldots
\end{itemize}


%-------------------------------------------------------------
\section{Recasting and reinterpretation}
%-------------------------------------------------------------

\subsection{Principles}

\subsection{Collaboration-internal recasting}

\begin{itemize}
    \item \href{https://github.com/recast-hep/recast-atlas}{recast-atlas}
        \begin{itemize}
            \item \NB{Currently have \texttt{recast-atlas} under collaboration-internal instead of public, as even though the code is open source you still need EOS access at CERN. So it is more ``CERN public'' than ``true public''.}
        \end{itemize}
    \item \ldots
\end{itemize}

\subsection{Public recasting tools}

\subsection{Differences in analyses}

\subsubsection{Measurements, Searches, and EFT Interpretations}

\subsubsection{Combinations and Global Analyses}

\subsection{Preservation of code and data products from theory studies}


%-------------------------------------------------------------
\section{Conclusions}
%-------------------------------------------------------------

\def\thefootnote{\fnsymbol{footnote}}
\setcounter{footnote}{0}

% Bibliography

\bibliographystyle{JHEP}
\bibliography{bib/preservation,bib/reinterpretation}

\end{document}
\newacronym{HEP}{HEP}{High Energy Physics}
\newacronym{CERN}{CERN}{European Organization for Nuclear Research}
\newacronym{LHC}{LHC}{Large Hadron Collider}
\newacronym{HL-LHC}{HL-LHC}{High-Luminosity \gls{LHC}}
\newacronym{EIC}{EIC}{Electron-Ion Collider}
\newacronym{RHIC}{RHIC}{Relativistic Heavy Ion Collider}
\newglossaryentry{ATLAS}{name={ATLAS}, description={A general-purpose detector at the \gls{LHC}}}
\newglossaryentry{CMS}{name={CMS}, description={A general-purpose detector at the \gls{LHC}}}
\newglossaryentry{CMSSW}{name={CMSSW}, description={Offline software for the \gls{CMS} collaboration}}
\newglossaryentry{ALICE}{name={ALICE}, description={A detector specialised in heavy-ion physics at the \gls{LHC}}}
\newglossaryentry{LHCb}{name={LHCb}, description={A detector specialised in $b$-physics at the \gls{LHC}}}
\newglossaryentry{STAR}{name={STAR}, description={One of four experiments at \gls{RHIC} studying quark-gluon plasma}}
\newglossaryentry{PHENIX}{name={PHENIX}, description={Pioneering High Energy Nuclear Interaction eXperiment, was a detector at \gls{RHIC} designed to investigate high energy collisions of heavy ions and protons}}
\newglossaryentry{sPHENIX}{name={sPHENIX}, description={A detector at \gls{RHIC} studying quark-gluon plasma by combining \gls{PHENIX} and \gls{STAR}}}
\newglossaryentry{LEP}{name={LEP}, description={Large Electron-Positron Collider}}
\newglossaryentry{DESY}{name={DESY}, description={German Electron Synchrotron}}
\newglossaryentry{HERA}{name={HERA}, description={Hadron-Electron Ring Accelerator at \gls{DESY}}}
\newglossaryentry{SLAC}{name={SLAC}, description={SLAC National Accelerator Laboratory}}
\newglossaryentry{BABAR}{name={BABAR}, description={A particle physics experiment at \gls{SLAC}}}
\newglossaryentry{LZ}{name={LZ}, description={LUX-ZEPLIN dark matter experiment}}
\newglossaryentry{DUNE}{name={DUNE}, description={Deep Underground Neutrino Experiment}}
\newacronym{HSF}{HSF}{\gls{HEP} Software Foundation}
\newacronym{IRIS-HEP}{IRIS-HEP}{Institute for Research and Innovation in Software for High Energy Physics}
\newglossaryentry{data}{name={data}, description={Data recorded from experiments that have been passed through the experiment's event reconstruction}}
\newglossaryentry{derived data}{name={derived data}, description={\Gls{data} that have passed through a size reduction step that prunes information, but which might also add additional calculated quantities to the data files.}}
\newglossaryentry{data product}{name={data product}, plural={data products}, description={All files containing selections of derived data and synthesised information from the various stages of an analysis. These might include summary plots and tables, histograms of kinematic distributions, fiducial cross sections, cross section limits, simplified model results, correlation information, analysis statistical workspace binaries or full statistical models, etc}}
\newglossaryentry{data preservation}{name={data preservation}, description={The procedures, practices, and standards of ensuring the long-term (i.e., decades beyond the end of an experiment) preservation, accessibility, and usability of data and derived data from experiments}}
\newglossaryentry{analysis preservation}{name={analysis preservation}, description={The procedures, practices, and standards of ensuring the long-term preservation, accessibility, and usability of information necessary to repeat an experimental analysis (starting from its associated preserved data) and generate all associated \glspl{data product}}}
\newglossaryentry{reinterpretation}{name={reinterpretation}, plural={reinterpretations}, description={Any type of new or updated interpretation of an experimental analysis or result, including the combination in, e.g., global fits or global averages.}}
\newglossaryentry{recasting}{name={recasting}, description={Reproducing the analysis logic in a simulation, considering a different physical process with a different phase space distribution, which might have different efficiencies and acceptances than the originally hypothesised model}}
\newacronym{FAIR}{FAIR}{Findable, Accessible, Interoperable, and Reusable}
\newacronym{BSM}{BSM}{Beyond-the-Standard Model}
\newacronym{QCD}{QCD}{Quantum Chromodynamics}
\newacronym{EFT}{EFT}{Effective Field Theory}
\newacronym{ONNX}{ONNX}{Open Neural Network Exchange}
\newacronym{ML}{ML}{machine learning}
\newacronym{DSL}{DSL}{domain specific language}
\newacronym{ADL}{ADL}{analysis description language}
\newacronym{BDT}{BDT}{boosted decision tree}
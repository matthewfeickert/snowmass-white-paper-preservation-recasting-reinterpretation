%   document style macros
\def\Title#1{\begin{center} {\LARGE #1 } \end{center}}
\def\Author#1{\begin{center}{ \sc #1} \end{center}}
\def\Address#1{\begin{center}{ \it #1} \end{center}}
\def\andauth{\begin{center}{and} \end{center}}
\def\submit#1{\begin{center}Submitted to {\sl #1} \end{center}}
\newcommand\pubblock{\rightline{\begin{tabular}{l} \pubnumber\\
         \pubdate \end{tabular}}}
\newenvironment{Abstract}{\begin{quotation} \begin{center}
                       ABSTRACT
     \end{center}\bigskip  }{\end{quotation}}
\newenvironment{Presented}{\begin{quotation} \begin{center} 
             CONTRIBUTED TO\end{center}\bigskip 
      \begin{center}\begin{large}}{\end{large}\end{center} \end{quotation}}
\def\submit#1{\begin{center}Submitted to {\sl #1} \end{center}}
\def\Acknowledgements{\bigskip  \bigskip \begin{center} \begin{large}
             \bf ACKNOWLEDGEMENTS \end{large}\end{center}}

\newcommand\snowmass{\begin{center}\rule[-0.2in]{\hsize}{0.01in}\\\rule{\hsize}{0.01in}\\
\vskip 0.1in Submitted to the  Proceedings of the US Community Study\\ 
on the Future of Particle Physics (Snowmass 2021)\\ 
\rule{\hsize}{0.01in}\\\rule[+0.2in]{\hsize}{0.01in} \end{center}}

%  personal abbreviations and macros

\newcommand{\hrulex}{%
  \vspace{1ex}
  \hrule
  \vspace{1ex}}

\newcommand\NB[1]{\textbf{N.B.}: {#1}}

\newcommand{\TODO}[2][]{\todo[inline,#1]{\textbf{TODO:} #2}}
\newcommand{\COMMENT}[2][]{\todo[inline,color=DarkOliveGreen1!90!Gray,#1]{#2}}
\newcommand{\ALERT}[2][]{\todo[inline,color=Red2!60!Orange!90,#1]{\textbf{!!:} #2}}
\newcommand{\QUESTION}[2][]{\todo[inline,color=SkyBlue1!90!Gray,#1]{\textbf{Q:} #2}}


\newcommand{\rivet}{Rivet\xspace} 
\newcommand{\contur}{Contur\xspace} 
\newcommand{\hepdata}{HEPData\xspace} 
\newcommand{\recast}{\textsc{Recast}\xspace} 
\newcommand{\madanalysis}{MadAnalysis5\xspace} 
\newcommand{\checkmate}{CheckMATE\xspace} 
\newcommand{\smodels}{SModelS\xspace} 
\newcommand{\gambit}{GAMBIT\xspace} 
\newcommand{\colliderbit}{ColliderBit\xspace} 
\newcommand{\simpleanalysis}{SimpleAnalysis\xspace} 
\newcommand{\delphes}{\textsc{Delphes}\xspace} 
